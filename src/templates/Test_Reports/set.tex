\documentclass{set}
\pagesettings

\title{
	{{sitename}}
}

\begin{document}
\maketitle
\clearpage
\heading{GURU NANAK DEV ENGINEERING COLLEGE, LUDHIANA}
{Accredited by NBA (AICTE), New Delhi (ISO 9001:2000 Certified)}
{Testing \& Consultancy Cell }
{tcc@gndec.ac.in }
{0161-2491193}
\sethead{List of Faculty/Experts} 
\listdata{Geotechnical}{Dr. J.N.Jha, Ph.D\\
Prof. Kulbir Singh Gill,M.E\\
Dr. B.S.Walia, Ph.D.\\
Prof. Harjinder Singh, M.E\\
Prof. Gurdeepak Singh, M.Tech.\\
}
\listdata{Structure}{Dr. Harpal Singh, Ph.D\\
Dr. Hardeep Singh Rai, Ph.D\\
Prof. Harvinder Singh, M.Tech\\
Dr. Jagbir Singh, Ph.D.\\
Prof. Kanwarjit Singh Bedi, M.Tech.\\
Prof. Parshant Garg, M.Tech\\
Prof. Harpreet Kaur, M.Tech.\\
Prof. Inderpreet Kaur, M.Tech.
}

\listdata{Highway}{Prof. Kulbir Singh Gill, M.E\\}
\listdata{Material Testing}{Dr. Jagbir Singh, Ph.D.\\
Prof. Kanwarjit Singh Bedi, M.Tech.\\
}

\listdata{Material Testing}{Dr. Jagbir Singh, Ph.D.\\
Prof. Kanwarjit Singh Bedi, M.Tech.\\
}

\listdata{Survey}{Dr. B.S.Walia, Ph.D.\\
}

\listdata{Chemical Testing}{Dr. R.P.Singh, Ph.D.\\
}

\listdata{Environmental Engg}{Prof. Puneet Pal Singh, M.E\\
}
\clearpage
\reportheading{SOIL INVESTIGATION REPORT}
\begin{enumerate}
\item{
\reportdetail{Date of Testing}{
	{{dateoftest}}
}}
\item{\reportdetail{Type of Structure}{
	{{strtype}}
}}
\item{\reportdetail{Site location}{ Latitude : 
{{lati}}
 Longitude : {{logitude}}
 }}

\item{\reportdetail{Tested in Presence of}{
{{p1}}\\
{{p2}}
}}
\item{\reportdetail{
	{{s1}}
}
{
	{{s2}}
}
}
\item{\reportdetail{Report Prepared by}{Dr. J. N. Jha\\
Prof. Kulbir Singh Gill\\
 Dr. B. S. Walia}}
\end{enumerate}
\clearpage
\section{Introduction}
The soil investigation for the proposed \textbf{
{{sitename}}
} had been taken up on request of \textbf{
{{s1}}
},\textbf{
{{s2}}
},\textbf{
{{s3}}
}. The field soil investigation as per requirements was carried out on \textbf{
{{dateoftest}}
} by
testing team of this institution in the presence of \textbf{
{{p1}}
} \& \textbf{
{{p2}}
}. Contractor of the concerned department.\par
The purpose of this soil investigation was to determine the nature of the subsoil stratum and the safe net
allowable bearing capacity of the soil.

\section{Field Soil Investigation}
Standard Penetration Test (S.P.T) was carried out at the proposed site for field soil investigation. The S.P.
Test was carried out as per I.S. Code 2131-1981 in the soil deposits at the foundation level or at an
interval of 1.5 m or at the location where change of soil strata takes place during the testing process. The
samples of the soil both disturbed and tube samples were collected at different depths and were properly
sealed in air-tight plastic bags after labelling them carefully to maintain the natural moisture content for
laboratory testing.
\section{Laboratory Testing}
The various samples (disturbed and tube) collected during field soil investigation were tested in the
laboratory (as per Standard Methods) for finding.
\begin{enumerate}
\item {Grain size analysis and wet analysis}
\item {Atterberg's limits}
\item{Field moisture content}
\item{Bulk density}
\item{Direct/triaxial shear/Unconfined compression tests}
\end{enumerate}
\section{Safe bearing capacity}
As per I.S. Code 6403-1981, the least of the following shall be taken as safe net allowable bearing
capacity of the soil.
\begin{enumerate}
\item{The safe net allowable bearing capacity from shear considerations is obtained by dividing net
ultimate bearing capacity by a suitable factor of safety.}
\item{The safe net allowable bearing pressure that can be imposed on the base of the foundation
without the settlement exceeding a permissible value is calculated either from settlement
analysis or from the Standard Penetration Test Values(N)whichever is applicable depending
upon the nature of sub soil strata.}
\end{enumerate}
\section{Underground Water Table}
The underground (i.e. sub-soil) water was encountered at a depth {{water}} m at the time of field soil
investigation.
\section{Proposed Structure or Substructure}
The  substructure i.e. foundation of the proposed {{strtype}} is taken in the form of circular raft of diameter {{diameter_B}} to be laid at a depth of   {{Depth_D}} m.  Thus for calculations of raft foundation following data is consi dered.\\

\reff{
	{{depth}}
}{
	{{diameter_B}}
}\\

The data obtained from the field soil investigation and the laboratory tests have been used in the
preparation of this report.
\section{Bearing Capacity Calculations}
\subsection{Bearing Capacity Based on Shear Considerations}
(As per I.S.Code - 6403:1981)\\

\reff{
	{{depth}}
}{
	{{diameter_B}}
}

The soil properties at the foundation level i.e. at {{depth}} m depth are:\\

\soilproperty{
	{{gama_g}}
}{
	{{C}}
}{
	{{phay}}
}{
	{{phayfe}}
}

Bearing Capacity factors are:\\

\bearingcapacity{
	{{nc}}
}{
	{{nq}}
}{
	{{ny}}
}

Shape factors are:\\

\shapefactor{1.3}{1.2}{0.6}

Depth factors are:\\    

\depth{
	{{dc}}
}{ 
	{{dqdy}}
}

Water table correction factor, w' = {{water}}\\

Ultimate net bearing capacity, $q_u$' = 0.67 x{{C}} x{{nc}} x1.3 x{{dc}} +{{gama_g}} x{{depth}} x(1 -{{nq}}) x1.2 x{{dqdy}} +0.5 x{{gama_g}} x{{diameter_B}} x{{ny}} x{{dqdy}} x0.6 x{{water}}\\
= {{pulse}} = {{eqtotal}} kN/$m^2$ \\	
Safe net allowable bearing capacity = $q_u'$/2.5 = {{eqtotal}}/2.5 = {{totaldy}} kN/$m^2$

\section{Bearing Capacity Based on Standard Penetration Test Value}
(As per I.S. Code -6403:1981) \\
\hspace{1.1cm}

\begin{tabularx}{\textwidth}{|*{6}{l|}}
\hline
 \multicolumn{1}{|m{1cm}|}{Sr No.} &\multicolumn{1}{m{2cm}|}{Depth(m)} &\multicolumn{1}{m{2.5cm}|}{Overburden pressure (kN/m\textsuperscript{2})}
&\multicolumn{1}{m{2cm}|}{Correction Factor} &\multicolumn{1}{m{2cm}|}{Observesd Value of N} &\multicolumn{1}{m{2.9cm}|}{Corrected Value of N}\\
\hline
{{table}}
\end{tabularx}

\hspace{1cm}

\reff{
	{{depth}}
}{
	{{diameter_B}}
}\\

\standardvalue{
	{{diameter_B}}
}{
	{{nvalue}}
}{
	{{S}}
}{
	{{water}}
}{
	{{value}}
}{
	nvalue
}
\clearpage
\section{Remarks:}

\remark{
	{{diameter_B}}
}{
	{{depth}}
}{
	{{nvalue}}
}{
	{{water}}
}

\footer{(Dr. B. S. Walia)\\
Associate Professor\\
Civil Engg. Department}{(Prof. Kulbir Singh Gill)\\
Associate Professor\\
Civil Engg. Department}

\footer{(Dr. J. N. Jha)\\
H.O.D., Civil Engg. Department}{(Dr. H. S. Rai)\\
Dean Testing \& Consultancy
    }
\end{document}